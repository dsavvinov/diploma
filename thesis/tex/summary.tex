\section*{Заключение}

В данной работе было рассмотрено расширение классических систем эффектов за счет введение условных эффектов. Были предложены правила комбинации эффектов, предусматривающие возможность масштабирования системы путем добавления новых операторов и эффектов.

Полученная система наилучшим образом проявила себя при работе с утверждениями, которые хорошо укладываются в классическое понятие <<эффекта>> -- т.е. некоторое изменение в контексте, производимое подпрограммой. Можно сделать вывод, что в систему, при необходимости, можно будет легко добавить большинство классических эффектов: \code{read}, \code{write}, \code{throws}, и др.

С другой стороны, введение понятий, слабо связанных с побочными эффектами, требует значительно б\a'{o}льших усилий и добавления новых конструкций и операторов. Так, умные приведения типа в коллекциях требуют формализма, который ближе к теории множеств, и потому укладываются в систему условных эффектов не очень хорошо. Тем не менее, в предложенной системе, технически, имеется возможность выражать и записывать такие понятия.

В качестве примера практической реализации, данная система была реализована в компиляторе языка \lang{Kotlin}. Для учета специфики данного языка (в частности, последовательных вычислений и их частичности), был введен специальный класс эффектов, описывающих исходы вычислений. Знание об этом классе заложено в систему для корректного извлечения эффектов некоторой последовательности вычислений, некоторые из которых завершаются аварийно (с исключением).

За счет добавления этой системы удалось улучшить статический анализ, выполняемый компилятором языка \lang{Kotlin}. В частности, были:

\begin{itemize}
  \item Поддержаны умные приведения типа, извлекающие информацию из выражений в условных конструкциях и включающих в себя вызовы функций.

  \item Поддержаны умные приведения типа, опирающиеся на факт (не)успешного завершения функции (например, в вызовах  \code{assert}).

  \item Поддержаны умные приведения типа в коллекциях (например, в вызовах вида \code{Collection<T>.filter})

  \item Улучшены существующие в компиляторе диагностики (например, об инициализации переменных) за счет эффектов, описывающих, что функция в ходе своего выполнения вызовет некоторую другую детерминированное число раз.
\end{itemize}

Также был реализован вывод эффектов из простых выражений.

В качестве возможных направлений развития следует отметить:

\begin{itemize}
  \item Расширение системы за счет добавления других конструкций логики: кванторы, импликации (в явном виде), предикаты

  \item Исследование других эффектов: например, связанных с многопоточностью, вводом-выводом, чтением-записью в переменные, и т.д.

  \item Улучшение системы автоматического вывода эффектов.
\end{itemize}
