\section{Введение}

Актуальность задачи статического анализа кода, на сегодняшний день, уже не вызывает ни у кого сомнения. Статический анализ обнаруживает ошибки в коде (как связанные с логикой, так и связанные с быстродействием, стилем), поставляет информацию, необходимую для проведения оптимизаций в коде, его рефакторинге. Статические анализаторы используются как в узкоспециализированных инструментах, основной задачей которых является непосредственный анализ кода, так и в качестве составной части других, более крупных и серьезных систем -- таких как компиляторы, IDE, оптимизаторы.

Одним из возможных методов статического анализа являются \term{системы эффектов}. Эффект -- это некоторое воздействие вычисления (функции, метода, подпрограммы) на состояние вычислителя. Простым примером эффекта является запись или чтение в разделяемую подпрограммами переменную, или же тот факт, что функция может бросить некоторое исключение (последний тип эффектов был реализован в языке \lang{Java} в виде механизма проверяемых исключений). Система эффектов отвечает за корректный учёт всех эффектов, их взаимодействие друг с другом, а также другие действия с ними (вывод, проверка, и т.д.)

Важным свойством, которому будет уделено значительно внимание, является то, что при анализе кода эффекты часто \textit{наблюдаются}. Так, в блоке \code{catch} в \lang{Java} мы наблюдаем тот эффект, что некоторая функция сгенерировала исключение. Однако в классических системах эффектов, это наблюдение не дает анализатору никакой информации. Вместе с тем, если бы было известно, какие условия вызывают такой эффект, то из факта наблюдения можно было бы сделать вывод, что окружение удовлетворяет поставленным условиям, тем самым извлекая больше информации из кода. 

В связи с этом, мы предлагаем систему эффектов, расширенную добавлением условных эффектов, про которые известны вызывающие их условия. Это позволяет описать в рамках системы более широкий набор функций и соответствующих им эффектов, и, что наиболее важно, извлечь из использования системы значительно больше пользы при тесной интеграции с компиляторами и/или IDE. 

В качестве демонстрации применения предложенной системы, будет изучена польза от рассмотрения условных эффектов на примере языка \lang{Kotlin}. Данный язык интересен из-за уникального механизма <<умных приведений типов>> (англ. \eng{smartcasts}) -- если в некотором участке кода статически гарантированно, что переменная имеет более частный тип, то компилятор сделает приведение к этому типу автоматически. Этот механизм естественным образом дополняет систему условных эффектов: пронаблюдав эффект и узнав некоторую типовую информацию о переменных в окружении, можно сделать умное приведение типов там, где раньше оно не делалось или же было вовсе невозможно. 

Таким образом, в работе будет рассмотрен ряд эффектов, естественным образом возникающих из предметной области -- в частности, эффекты, сообщающие о типе или значении переменных окружения. Кроме того, в работе будет показано, как ввести эффекты, сообщающие о том, что некоторая процедура была вызвана детерминированное количество раз. Данные эффекты могут быть очень полезны в комбинации с другими видами анализа, нуждающимися в подобных гарантиях для сохранения консервативности (например, анализ инициализации переменных).

Также будут рассмотрены вопросы комбинации условных эффектов при вложенных вызовах. Отдельное внимание будет уделено проблеме работы с эффектами в императивном языке, вносящим такие понятия, как <<последовательность исполнения>> и <<успешное/не успешное завершение вычисления>>.

В заключение, будет произведен анализ применимости систем условных эффектов для различных видов прикладных задач, будут намечены направления развития данного исследования.

