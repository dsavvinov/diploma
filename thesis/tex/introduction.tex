\section{Введение}

Актуальность задачи статического анализа кода, на сегодняшний день, уже не вызывает ни у кого сомнения. Статический анализ обнаруживает ошибки в коде (как связанные с логикой, так и связанные с быстродействием, стилем), поставляет информацию, необходимую для проведения оптимизаций в коде, его рефакторинге. Статические анализаторы используются как в узкоспециализированных инструментах, основной задачей которых является непосредственный анализ кода, так и в качестве части других, более крупных и серьезных систем -- таких как компиляторы, IDE, оптимизаторы.

Одной из наиболее важных и сложных задач в этой области является т.н. \definition{межпроцедурный} (англ. \textit{interprocedural}) анализ, т.е. анализ, выполняющийся над всей программой в целом (в противовес \definition{внутрипроцедурному} анализу, который, как очевидно из названия, выполняется ровно над одной отдельно взятой процедурой). Он позволяет извлекать более сложные и подробные утверждения о коде, но при этом является более трудоемким как с вычислительной точки зрения, так и с точки зрения реализации.

Одними из возможных методов межпроцедурного анализа являются \definition{системы эффектов}. Эффект -- это некоторое воздействие вычисления (функции, метода, подпрограммы) на состояние вычислителя. Простым примером эффекта является запись или чтение в разделяемую подпрограммами переменную, или же тот факт, что функция может бросить некоторое исключение (последний тип эффектов был реализован в языке \lang{Java} в виде механизма проверяемых исключений). Система эффектов отвечает за корректный учёт всех эффектов, их взаимодействие друг с другом, а также другие действия с ними (вывод, проверка, и т.д.)

Наличие аннотаций эффектов на функциях в программе может значительно упростить межпроцедурный анализ за счет того, что эффекты являются очень краткой и емкой формой описания межпроцедурного взаимодействия. Анализатору, встретившему вызов подпрограммы, для которой известны ее эффекты, не обязательно начинать анализ самой подпрограммы -- ее поведение в достаточном объеме описано перечислением ее эффектов. Нужно лишь уметь корректно применить эту информацию в конкретном месте вызове. 

Следует отметить, что при анализе кода эффекты часто \textit{наблюдаются}. Так, в блоке \code{catch} в \lang{Java} мы наблюдаем тот эффект, что некоторая подпрограмма сгенерировала исключение. Знание о том, какие условия вызвали этот эффект, могло бы дать больше информации об окружении, открывая новые возможности для анализа.

В связи с этом, мы предлагаем систему эффектов, расширенную добавлением условных эффектов, про которые известны вызывающие их условия. Это позволяет описать в рамках системы гораздо более широкий набор различных функций и соответствующих им эффектов, и, что наиболее важно, извлечь из использования системы значительно больше пользы при тесной интеграции с компиляторами и/или IDE. 

В работе будут рассмотрены вопросы комбинации условных эффектов при вложенных вызовах. Отдельное внимание будет уделено проблеме работы с эффектами в императивном языке, вносящим такие понятия, как <<последовательность исполнения>> и <<успешное/не успешное завершение вычисления>>.

В качестве демонстрации применения предложенной системы, будет изучена польза от рассмотрения условных эффектов на примере языка \lang{Kotlin}. Данный язык интересен из-за уникального механизма <<умных приведений типов>> (англ. \emph{smartcasts}) -- если в некотором участке кода статически гарантированно, что переменная имеет более частный тип, то компилятор сделает приведение к этому типу автоматически. Этот механизм естественным образом дополняет систему условных эффектов: пронаблюдав эффект и узнав некоторую типовую информацию о переменных в окружении, можно сделать умное приведение типов там, где раньше оно не делалось или же было вовсе невозможно. 

Также в работе будут рассмотрены эффекты, сообщающие о том, что некоторая процедура была вызвана детерминированное количество раз, что очень полезно в комбинации с другими видами анализа, нуждающимися в подобных гарантиях для сохранения консервативности (например, анализ инициализации переменных).

В заключение, будет произведен некоторый анализ применимости систем условных эффектов для различных видов прикладных задач, будут намечены направления развития данного исследования.

