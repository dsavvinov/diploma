\section{Анализ полученной системы}

В данной главе мы постараемся дать некоторую оценку полученной системе, описать ее сильные и слабые стороны, а также очертить границы ее применимости на практике. 

Итак, в данной работе были рассмотрены системы эффектов. Классические эффекты не имеют прикрепленных к ним условий, сообщающих, когда данный эффект будет сгенерирован. Это ведет или к тому, что некоторые функции становится невозможно проаннотировать (например, многие функции возвращают \code{null} только если переданный им аргумент был \code{null}, и такие функции невозможно аннотировать \code{@NotNull}), или же к ограничению свободы анализатора (к примеру, аннотация \code{throws} в \lang{Java} означает, что функция \emph{может} бросить исключение, поэтому отсутствие исключения не дает никакой информации анализатору).

В данной работе была предпринята попытка убрать это ограничение классических эффектов за счет введения т.н. условных эффектов -- т.е. эффектов, к которым прикреплено вызывающее их условие. Это позволяет, с одной стороны, сделать эффекты более строгими -- теперь аннотация \code{Throws} означает, что функция \emph{бросила} исключение, а не \emph{могла} бросить. Более строгие эффекты развязывают руки анализатору, т.к. содержат в себе больше информации -- например, отсутствие исключения позволяет сделать вывод, что соответствующие условие не было выполнено. 

С другой же стороны, введение условий позволяет компенсировать излишнюю строгость самих эффектов -- было бы странно часто видеть функции, для которых просто так верна строгая формулировка аннотации \code{Throws}, в то время как есть вполне жизненная и полезная функция \code{assert}, для которой верна формулировка \code{condition == false -> Throws AssertionError}.

Система условных эффектов естественным образом наследует полезные свойства систем эффектов как таковых: возможность ручной аннотации, в том числе и со стороны пользователя; отсутствие обязательных проверок времени исполнения. Очень важным свойством является то, что вычислительная сложность анализа зависит от количества и типов эффектов, а не от длины кода. Это обеспечивает большую гибкость системы с точки зрения пользователя, позволяя ему самому выбирать между более полным, но медленным анализом, и легковесными аннотациями, не требующими много времени. 

Разработанная система слабо привязана к конкретным эффектам и операторам, что делает ее более гибкой и позволяет легче подстроиться под то или иное конкретное применение. Так, в данной работе в качестве иллюстрации было рассмотрен некоторое малое подмножество операторов языка \lang{Kotlin} и небольшое количество эффектов, необходимых для решения некоторых частных проблем. Пользуясь описанными в работе подходами, можно добавить и другие операторы и эффекты, причем это потребует от минимального до вообще никакого вмешательства в базовые алгоритмы системы (связанные с комбинированием, выводом информации и т.д.). 

В предложенной системе были подробно рассмотрены проблемы вычисления эффектов в присутствии частичных вычислений, что очень актуально для применения подобных систем на практике. Для решения этой проблемы пришлось все же заложить априорное знание о некотором классе эффектов, которые сообщают об успешности или неуспешности вычисления и дают анализатору информацию, необходимую для корректной комбинации эффектов. 

Также не были оставлены без внимания и другие проблемы чисто практического характера. Например, весьма характерной проблемой для условных эффектов является комбинаторный взрыв размера аннотаций при последовательном комбинировании. Для решения этой проблемы в данной работе было намечено два принципиально разных подхода -- первый уменьшает размер схемы без потери информации и основывается на сокращении логических формул. Второй подход был назван аппроксимацией схем и позволяет добиться более сильного уменьшения размера, платя за это потерей информации о некоторых эффектах.

Следует отметить, что как и любой инструмент, система условных эффектов требует соответствующего окружения, способного воспользоваться предоставляемыми ею возможностями. Так, при наблюдении эффектов, система может выводить, при каких условиях на контекст такие эффекты могли быть сгенерированы, получая весьма емкую и ценную информацию об окружении. В данной работе в качестве окружения использовался компилятор языка \lang{Kotlin}, в которым такой информации находилось естественное применение в механизме умных приведений типов.

Кроме того, простота добавления новых синтаксических конструкций не говорит о том, что в систему легко добавить новые понятия в целом. Так, было замечено, что утверждения про коллекции тесно связаны с формализмом теории множеств, и поддержка таких утверждений в системе требует введения ряда дополнительных операторов. Каждый из подобных операторов вводится концептуально просто, но требует некоторой нетривиальной технической работы, тем самым делая добавление одних понятий (не требующих введения новых конструкций) более удобным, чем других. В частности, легче всего добавляются понятия, являющиеся <<побочными эффектами>> в полном смысле этого слова -- например, эффекты вызова другой функции, чтения, записи, и т.д. Чем дальше вводимая абстракция от <<побочного эффекта>>, тем больших усилий она требует для формализации в рамках данной системы.


Таким образом, с точки зрения научной новизны, данное исследование постаралось внести вклад в изучение следующих вопросов:

\begin{itemize}
	\item Условные эффекты с достаточно богатыми условиями
	
	\item Исчисление эффектов в присутствии частичных вычислений
	
	\item Борьба с экспоненциальным ростом размера аннотаций
\end{itemize}


В качестве основных преимуществ полученной системы следует отметить:

\begin{itemize}
	\item Сложность анализа зависит от количества и типов аннотаций, а не от дины кода
	
	\item Расширяемость системы по базовым конструкциям (операторам и эффектам)
	
	\item Гибкость и интенсивность анализа, достигнутая за счет использования условных эффектов
	
	\item Практичность, достигнутая за счет рассмотрения ряда технических вопросов (роста длины аннотаций) и отказа от слишком теоретичных предположений (тотальности вычислений)
\end{itemize}

Недостатками системы являются: 

\begin{itemize}
	\item Недостаточная формализованность, которая может послужить препятствием для понимания, реализации или расширения данной работы
	
	\item Необходимость в использовании дополнительных инструментов и методик для получения практических результатов (например, в данной работе в качестве такого инструмента использовался анализатор компилятора \lang{Kotlin}, выполняющий анализ потока данных)
	
	\item Затруднения при формализации некоторых понятий, не укладывающихся в концепцию эффекта (например, утверждения про свойства коллекции)
\end{itemize}
   
Таким образом, наилучшее применение данная система находит в языках программирования, в качестве побочного инструмента анализа, приходящего на помощь основному в некоторых специальных случаях. В языках программирования наиболее полно раскрываются такие положительные свойства систем эффектов, как возможность ручной аннотации и гибкость в плане затрачиваемых ресурсов, позволяя пользователям языка подстраивать систему под своим нужды.

Разработанная система не должна рассматриваться как средство формальной спецификации (по типу языков спецификаций или даже парадигмы \code{design-by-contract}), несмотря на то, что аннотации эффектов с условиями могут ложно натолкнуть на такую мысль. Для этого данной системе не хватает формальности и выразительности, которые были аккуратно пожертвованы в угоду легковесности и прозрачности.
