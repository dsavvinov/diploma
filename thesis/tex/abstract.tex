\section*{Аннотация}

На сегодняшний день, статический анализ кода является неотъемлемой частью процесса разработки программного обеспечения, выполняя ряд важнейших функций: от тривиальной автоматизации рутинных действий (например, приведения типов) и поиска простых ошибок, до сложнейших оптимизаций, значительно влияющих на производительность программы. 

Вместе с тем, большинство существующих систем либо не производят межпроцедурного анализа вовсе, либо производят в ограниченном количестве, что серьезно ограничивает полноту результирующей диагностики.

Данная работа посвящена изучению одного из возможных подходов к межпроцедурному анализу, основанного на учете т.н. <<побочных эффектов>> (англ. \eng{side-effects}) функций. Была предложена новая система описания побочных эффектов, основанная на <<схемах эффектов>> -- кратких описаний всех возможных побочных эффектов функции и вызывающих их условий. В качестве демонстрации возможностей предложенной системы, ее прототип был реализован на практике и успешно использован для улучшения статического анализа в языке \lang{Kotlin}.