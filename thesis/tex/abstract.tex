\section*{Аннотация}

На сегодняшний день, статический анализ кода выполняет ряд важнейших функций: автоматизация рутинных действий (таких как приведение типов), поиск простых ошибок, оптимизации кода, рефакторинг.

Статический анализ можно условно разделить на внутрипроцедурный (работающий только над отдельно взятой процедурой) и на межпроцедурный (рассматривающий при анализе всю программу в целом). Вместе с тем, большинство существующих систем не выполняют интенсивного межпроцедурного анализа, что серьезно ограничивает полноту результирующей диагностики.

Данная работа посвящена изучению одного из возможных подходов к межпроцедурному анализу, основанного на учете т.н. <<побочных эффектов>> (англ. \eng{side-effects}) функций. Значительное внимание было уделено <<условным эффектам>> -- т.е. эффектам, которые вызываются только при исполнении некоторых условий. Было введено понятие <<схемы эффектов>> -- краткого перечисления всех возможных условных эффектов функции. На основе этих понятий была разработана система условных эффектов.

В качестве демонстрации возможностей предложенной системы, ее прототип был реализован на практике и успешно использован для улучшения статического анализа в языке \lang{Kotlin}.