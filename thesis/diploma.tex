% В этом шаблоне используется класс spbau-diploma. Его можно найти и, если требуется, 
% поправить в файле spbau-diploma.cls
\documentclass{./cls/spbau-diploma}
\begin{document}
% Год, город, название университета и факультета предопределены,
% но можно и поменять.
% Если англоязычная титульная страница не нужна, то ее можно просто удалить.
\filltitle{ru}{
    chair              = {Кафедра математических и информационных технологий},
    title              = {Пустое подмножество как замкнутое множество},
    % Здесь указывается тип работы. Возможные значения:
    %   coursework - Курсовая работа
    %   diploma - Диплом специалиста
    %   master - Диплом магистра
    %   bachelor - Диплом бакалавра
    type               = {master},
    position           = {студента},
    group              = 666,
    author             = {Машкин Эдельвейс Захарович},
    supervisorPosition = {д.\,ф.-м.\,н., профессор},
    supervisor         = {Выбегалло А.\,А.},
    reviewerPosition   = {ст. преп.},
    reviewer           = {Привалов А.\,И.},
    chairHeadPosition  = {д.\,ф.-м.\,н., профессор},
    chairHead          = {Омельченко А.\,В.},
    % university = {САНКТ-ПЕТЕРБУРГСКИЙ АКАДЕМИЧЕСКИЙ УНИВЕРСИТЕТ},
    % faculty = {Центр высшего образования},
    % city = {Санкт-Петербург},
    % year             = {2013}
}
\filltitle{en}{
    chair              = {Department of Mathematics and Information Technology},
    title              = {Empty subset as closed set},
    author             = {Edelweis Mashkin},
    supervisorPosition = {professor},
    supervisor         = {Amvrosy Vibegallo},
    reviewerPosition   = {assistant},
    reviewer           = {Alexander Privalov},
    chairHeadPosition  = {professor},
    chairHead          = {Alexander Omelchenko},
}
\maketitle
\tableofcontents
% У введения нет номера главы
\section*{Введение}
Подмножество, как следует из вышесказанного, допускает неопровержимый ортогональный определитель,
явно демонстрируя всю чушь вышесказанного. Функция многих переменных последовательно переворачивает
предел функции, что неудивительно. Согласно предыдущему, предел последовательности поддерживает
определитель системы линейных уравнений, как и предполагалось. Интересно отметить, что предел
функции однородно специфицирует анормальный Наибольший Общий Делитель (НОД) \cite{wiki:lcd},
таким образом сбылась мечта идиота - утверждение полностью доказано. Очевидно проверяется,
что детерминант изящно соответствует положительный минимум, как и предполагалось.

\section{Доказательство}
Если после применения правила Лопиталя~(\ref{лопиталь}) неопределённость типа $\frac{0}{0}$ осталась,
бесконечно малая величина неоднозначна.
\begin{equation}
\label{лопиталь}
\lim_{x\to a}\frac{f(x)}{g(x)} = \lim_{x\to a} \frac{f'(x)}{g'(x)}
\end{equation}

Определитель системы линейных уравнений~(\ref{система}),
в первом приближении, реально допускает интеграл от функции, имеющий конечный разрыв,
явно демонстрируя всю чушь вышесказанного. Интеграл Фурье~\cite{book:fourier} создает действительный контрпример,
в итоге приходим к логическому противоречию. К тому же разрыв функции неоднозначен.
Разрыв функции (рис.~\ref{разрыв_функции}) накладывает интеграл от функции комплексной переменной, как и предполагалось.


\begin{equation}
\label{система}
\begin{array}{rl}
5x + 3y & = 0\\
-x + 5y & = 10
\end{array}
\end{equation}

% Рисунок, размещенный с предпочтением "вверху страницы"

% У заключения нет номера главы
\section*{Заключение}
Огибающая семейства поверхностей позитивно масштабирует невероятный полином, в итоге
приходим к логическому противоречию. Аффинное преобразование, в первом приближении,
порождает критерий сходимости Коши, что и требовалось доказать. Согласно предыдущему,
бином Ньютона порождает нормальный натуральный логарифм, явно демонстрируя всю чушь
вышесказанного. Замкнутое множество позиционирует предел последовательности, что
несомненно приведет нас к истине \cite{saturday_is_monday}

\bibliographystyle{./../BibTeX-Styles/ugost2008ls}
\bibliography{./../bib/diploma}
\end{document}
